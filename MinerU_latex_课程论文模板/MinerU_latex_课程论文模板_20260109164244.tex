% This LaTeX document needs to be compiled with XeLaTeX.
\documentclass[
  10
]{article}
\usepackage[margin=2cm,includehead=true,includefoot=true,centering,]{geometry}
\usepackage{xcolor}
\usepackage{ucharclasses}
\usepackage{hyperref}
\usepackage{amsmath,amssymb}
\usepackage{amsfonts}
\usepackage[version=4]{mhchem}
\usepackage{stmaryrd}
\usepackage{bbold}
\usepackage{polyglossia}
\usepackage{fontspec}
\usepackage{ctex}
\usepackage[export]{adjustbox}
\usepackage{tabularx}
\usepackage{booktabs}

\usepackage{setspace}
\setstretch{1.2}

\makeatletter
\@ifundefined{KOMAClassName}{% if non-KOMA class
  \IfFileExists{parskip.sty}{%
    \usepackage{parskip}
  }{% else
    \setlength{\parindent}{0pt}
    \setlength{\parskip}{6pt plus 2pt minus 1pt}}
}{% if KOMA class
  \KOMAoptions{parskip=half}}
\makeatother
\usepackage{longtable,booktabs,array}
\newcounter{none} % for unnumbered tables
\usepackage{calc} % for calculating minipage widths
% Correct order of tables after \paragraph or \subparagraph
\usepackage{etoolbox}
\makeatletter
\patchcmd\longtable{\par}{\if@noskipsec\mbox{}\fi\par}{}{}
\makeatother
% Allow footnotes in longtable head/foot
\IfFileExists{footnotehyper.sty}{\usepackage{footnotehyper}}{\usepackage{footnote}}
\makesavenoteenv{longtable}
\setlength{\emergencystretch}{3em} % prevent overfull lines
\providecommand{\tightlist}{%
  \setlength{\itemsep}{0pt}\setlength{\parskip}{0pt}}


\hypersetup{colorlinks=true, linkcolor=blue, filecolor=magenta, urlcolor=cyan,}
\urlstyle{same}

\usepackage{colortbl}
\definecolor{table-row-color}{HTML}{999999}
\definecolor{table-rule-color}{HTML}{999999}
%\arrayrulecolor{black!40}
\arrayrulecolor{table-rule-color}     % color of \toprule, \midrule, \bottomrule
\setlength{\aboverulesep}{0pt}
\setlength{\belowrulesep}{0pt}


\setotherlanguages{english}
\IfFontExistsTF{Source Han Serif CN}
{\newfontfamily\chinesefont{Source Han Serif CN}}
{\IfFontExistsTF{Noto Serif CJK SC}
  {\newfontfamily\chinesefont{Noto Serif CJK SC}}
  {\IfFontExistsTF{SimSun}
    {\newfontfamily\chinesefont{SimSun}}
    {\IfFontExistsTF{FangSong}
      {\newfontfamily\chinesefont{FangSong}}
      {\newfontfamily\chinesefont{Arial Unicode MS}}
}}}
\IfFontExistsTF{Times New Roman}
{\newfontfamily\englishfont{Times New Roman}}
{\IfFontExistsTF{Liberation Serif}
  {\newfontfamily\englishfont{Liberation Serif}}
  {\IfFontExistsTF{DejaVu Serif}
    {\newfontfamily\englishfont{DejaVu Serif}}
    {\newfontfamily\englishfont{Arial}}
}}


\author{}
\date{}

\begin{document}


\section{研究生课程论文}\label{ux7814ux7a76ux751fux8bfeux7a0bux8bbaux6587}

\section{课程论文名称}\label{ux8bfeux7a0bux8bbaux6587ux540dux79f0}

研究生:李XX

提交日期:202X年XX月XX日

研究生签名:

{\def\LTcaptype{none} % do not increment counter
\begin{longtable}[]{@{}|l|l|l|l|@{}}
\toprule\noalign{}
\endhead
\bottomrule\noalign{}
\endlastfoot
\hline
学号 & 2025XXXXXXXXX & 学院 & XX学院 \\
\hline
课程编号 & S0001001 & 课程名称 & XX \\
\hline
学位类别 & □博士 □硕士 & 任课教师 & 张XX \\
\hline
学习形式 & □全日制 □非全日制 & 学期 & 2025-2026学年第一学期 \\
\hline
\multicolumn{4}{@{}l@{}}{%
教师评语:} \\
\hline
\multicolumn{4}{@{}l@{}}{%
成绩评定: 分 任课教师签名: 年 月 日} \\
\hline
\end{longtable}
}

\section{说明}\label{ux8bf4ux660e}

1、课程论文要有题目、作者姓名、摘要、关键词、正文及参考文献。论文题目由研究生结合课程所学内容选定;摘要500字以下,博士生课程论文要求有英文摘要;关键词3\textasciitilde5个;参考文献不少于10篇,并应有一定的外文文献。

2、课程论文须由本人独立撰写,不得抄袭、剽窃、代写,不得使用人工智能工具或软件生成内容。若发现存在上述行为,按作弊处理,本门课程考核成绩计0分。

3、课程论文用A4纸双面打印。字体全部用宋体简体,题目要求用小二号字加粗,标题行要求用小四号字加粗,正文内容要求用小四号字;经学院同意,课程论文可以用英文撰写,字体全部用Times
New
Roman,题目要求用18号字加粗;标题行要求用14号字加粗,正文内容要求用12号字;行距为2倍行距(方便教师批注);页边距左为
\(3\mathrm{cm}\) 、右为 \(2\mathrm{cm}\) 、上为 \(2.5\mathrm{cm}\)
、下为 \(2.5\mathrm{cm}\) ;其它格式请参照学位论文要求。

4、学位类别按博士、硕士填写,学习形式按全日制、非全日制填写。

5、篇幅、内容等由任课教师提出具体要求。

\section{硕士课程论文}\label{ux7855ux58ebux8bfeux7a0bux8bbaux6587}

\section{中文题目
(宋体,小二号字,加粗)}\label{ux4e2dux6587ux9898ux76ee-ux5b8bux4f53ux5c0fux4e8cux53f7ux5b57ux52a0ux7c97}

研究生姓名 (宋体,四号字,加粗)

摘要 (小四号字加粗) : \(\times   \times   \times\) (小四号字)

关键词 (小四号字加粗) : \(\times \times ; \times \times ; \cdots\)
(小四号字)

正文部分(标题行用小四号字加粗,正文内容用小四号字)

参考文献 (小四号字加粗)

{[}1{]} \(\times \times \times\) (五号字)

注:正式上交课程论文时,请删除蓝色字体内容

\section{博士课程论文}\label{ux535aux58ebux8bfeux7a0bux8bbaux6587}

\section{中文题目
(宋体,小二号字,加粗)}\label{ux4e2dux6587ux9898ux76ee-ux5b8bux4f53ux5c0fux4e8cux53f7ux5b57ux52a0ux7c97-1}

研究生姓名 (宋体,四号字,加粗)

摘要(小四号字加粗): \(\times \times x\) (小四号字)

关键词(小四号字加粗): \(\times \times\) ; \(\times \times\)
;\ldots(小四号字)

正文部分(标题行用小四号字加粗,正文内容用小四号字)

参考文献 (小四号字加粗)

{[}1{]} \(\times \times \times\) (五号字)

\section{英文题目 (Times New Roman, 18
号字,加粗)}\label{ux82f1ux6587ux9898ux76ee-times-new-roman-18-ux53f7ux5b57ux52a0ux7c97}

研究生姓名拼音 (Times New Roman, 14 号字,加粗)

Abstract (12号字加粗) : \(\times \times \times\) (12号字)

Key words (12 号字加粗) :
\(\times   \times  , \times   \times  ,\ldots\) (12 号字)

注:正式上交课程论文时,请删除蓝色字体内容

\end{document}
