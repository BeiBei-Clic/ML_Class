% This LaTeX document needs to be compiled with XeLaTeX.
%!mode::"TeX:UTF-8"
% !TEX program = xelatex
\documentclass[
  10
]{article}
\usepackage[margin=2cm,includehead=true,includefoot=true,centering,]{geometry}
\usepackage{xcolor}
\usepackage{ucharclasses}
\usepackage{hyperref}
\usepackage{amsmath,amssymb}
\usepackage{amsfonts}
\usepackage[version=4]{mhchem}
\usepackage{bbold}
\usepackage{polyglossia}
\usepackage{fontspec}
\usepackage[fontset=none]{ctex}
\usepackage[export]{adjustbox}
\usepackage{tabularx}
\usepackage{booktabs}

\usepackage{setspace}
\setstretch{1.2}

\makeatletter
\@ifundefined{KOMAClassName}{% if non-KOMA class
  \IfFileExists{parskip.sty}{%
    \usepackage{parskip}
  }{% else
    \setlength{\parindent}{0pt}
    \setlength{\parskip}{6pt plus 2pt minus 1pt}}
}{% if KOMA class
  \KOMAoptions{parskip=half}}
\makeatother
\usepackage{longtable,booktabs,array}
\newcounter{none} % for unnumbered tables
\usepackage{calc} % for calculating minipage widths
% Correct order of tables after \paragraph or \subparagraph
\usepackage{etoolbox}
\makeatletter
\patchcmd\longtable{\par}{\if@noskipsec\mbox{}\fi\par}{}{}
\makeatother
% Allow footnotes in longtable head/foot
\IfFileExists{footnotehyper.sty}{\usepackage{footnotehyper}}{\usepackage{footnote}}
\makesavenoteenv{longtable}
\setlength{\emergencystretch}{3em} % prevent overfull lines
\providecommand{\tightlist}{%
  \setlength{\itemsep}{0pt}\setlength{\parskip}{0pt}}

\hypersetup{colorlinks=true, linkcolor=blue, filecolor=magenta, urlcolor=cyan,}
\urlstyle{same}

\usepackage{colortbl}
\definecolor{table-row-color}{HTML}{999999}
\definecolor{table-rule-color}{HTML}{999999}
%\arrayrulecolor{black!40}
\arrayrulecolor{table-rule-color}     % color of \toprule, \midrule, \bottomrule
\setlength{\aboverulesep}{0pt}
\setlength{\belowrulesep}{0pt}

\setotherlanguages{english}
\IfFontExistsTF{Times New Roman}
{\setmainfont{Times New Roman}}
{\IfFontExistsTF{Times}
  {\setmainfont{Times}}
  {\IfFontExistsTF{Liberation Serif}
    {\setmainfont{Liberation Serif}}
    {\setmainfont{Times New Roman}}
}}
\newcommand*{\CJKMainFontName}{Songti SC Regular}
\IfFontExistsTF{Songti SC Regular}{}{%
  \IfFontExistsTF{Songti SC}{\renewcommand*{\CJKMainFontName}{Songti SC}}{%
    \IfFontExistsTF{STSongti-SC-Regular}{\renewcommand*{\CJKMainFontName}{STSongti-SC-Regular}}{%
      \IfFontExistsTF{STSong}{\renewcommand*{\CJKMainFontName}{STSong}}{%
        \IfFontExistsTF{FandolSong-Regular}{\renewcommand*{\CJKMainFontName}{FandolSong-Regular}}{}%
      }%
    }%
  }%
}
\setCJKmainfont{\CJKMainFontName}
\newcommand*{\CJKSansFontName}{Heiti SC Light}
\IfFontExistsTF{Heiti SC Light}{}{%
  \IfFontExistsTF{STHeitiSC-Light}{\renewcommand*{\CJKSansFontName}{STHeitiSC-Light}}{%
    \IfFontExistsTF{STHeiti Light}{\renewcommand*{\CJKSansFontName}{STHeiti Light}}{%
      \IfFontExistsTF{Heiti SC}{\renewcommand*{\CJKSansFontName}{Heiti SC}}{%
        \IfFontExistsTF{STHeiti}{\renewcommand*{\CJKSansFontName}{STHeiti}}{%
          \IfFontExistsTF{FandolHei-Regular}{\renewcommand*{\CJKSansFontName}{FandolHei-Regular}}{}%
        }%
      }%
    }%
  }%
}
\setCJKsansfont{\CJKSansFontName}
\setCJKfamilyfont{hei}{\CJKSansFontName}
\setCJKfamilyfont{song}{\CJKMainFontName}
\providecommand{\heiti}{\CJKfamily{hei}}

\author{}
\date{}

\begin{document}

%----------------------------------------------------------------------------------------
%   Cover
%----------------------------------------------------------------------------------------
\thispagestyle{empty}
\begin{figure}[ht]
\centering
\includegraphics[height=2.75cm]{figures/title.png}
\end{figure}
\vspace{1.5cm}

\begin{center}
{\zihao{2}\bfseries 华南理工大学\par}
\vspace{0.6cm}
{\zihao{3}\bfseries 课程论文名称\par}
\vspace{0.8cm}
{\zihao{4}研究生:李XX\par}
\end{center}
\vspace{0.2cm}
{\zihao{5}%
\noindent\begin{tabular*}{\linewidth}{@{}l@{\extracolsep{\fill}}l@{}}
提交日期:\hspace{4em}年\hspace{1em}月\hspace{1em}日 & 研究生签名:\hspace{6em} \\
\end{tabular*}
}
\vspace{0.2cm}

{\zihao{5}%
\def\LTcaptype{none} % do not increment counter
\begin{longtable}[]{@{}|l|l|l|l|@{}}
\toprule\noalign{}
\endhead
\bottomrule\noalign{}
\endlastfoot
\hline
学号 & 2025XXXXXXXXX & 学院 & XX学院 \\
\hline
课程编号 & S0001001 & 课程名称 & XX \\
\hline
学位类别 & □博士 □硕士 & 任课教师 & 张XX \\
\hline
学习形式 & □全日制 □非全日制 & 学期 & 2025-2026学年第一学期 \\
\hline
\multicolumn{4}{|p{0.93\linewidth}|}{%
\textbf{教师评语:}\par\vspace{6cm}} \\
\hline
\multicolumn{4}{|l|}{%
成绩评定: 分 任课教师签名: 年 月 日} \\
\hline
\end{longtable}
}
\pagebreak[4]


\section{Title and Abstract}
\textbf{Title:} 在此填写论文题目(简洁、可检索、包含核心概念)。

\textbf{Authors:} 在此填写作者信息(匿名投稿时可隐藏)。

\textbf{Abstract:} 用一段话覆盖问题、方法、结果与意义,控制在 150--250 词。

\textbf{Keywords:} 3--5 个关键词。

\section{Introduction}
动机与问题背景、研究空白、关键直觉与贡献点。建议在文末用项目符号列出 2--4 条贡献:
\begin{itemize}
  \item 贡献 1:描述核心方法或视角。
  \item 贡献 2:描述关键实验或理论结果。
  \item 贡献 3:描述可复现性或新基准。
\end{itemize}

\section{Related Work}
按与你的方法、设定或评测最相关的维度分组讨论,不做流水账;最后点明你的差异与必要性。

\section{Background / Preliminaries}
当依赖新设定或符号体系时提供必要背景;否则可合并至方法部分。

\section{Method}
建议结构:问题定义 $\rightarrow$ 模型/算法 $\rightarrow$ 训练或求解 $\rightarrow$ 推理与复杂度。
可以配一张整体框图与算法伪代码,突出影响性能与稳定性的实现细节。

\section{Theory / Analysis}
给出关键定理、假设条件与证明思路;完整证明可放附录。

\section{Experiments}
\textbf{Setup:} 数据集、基线、指标、实现细节与算力配置。

\textbf{Main Results:} 主表与统计显著性(若适用)。

\textbf{Ablations:} 模块、超参或数据量消融。

\textbf{Efficiency:} 时间、显存与吞吐。

\textbf{Qualitative:} 可视化与失败案例(可选)。

\section{Discussion}
讨论机制解释、适用边界、可能失败场景与与已有方法的联系。

\section{Limitations}
单列关键假设与潜在失效情形,避免被 reviewer 在 Checklist 中追问。

\section{Conclusion}
简短总结贡献与结果,并指出下一步工作方向。

\section*{References}
参考文献放在此处,不少于课程要求数量。

\section*{NeurIPS Paper Checklist}
按官方 Checklist 填写(需要随主文同一 PDF 提交)。

\appendix
\section{Appendix}
放置完整证明、更多实验、实现细节与附加可视化。

\end{document}
