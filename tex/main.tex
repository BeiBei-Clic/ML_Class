%!mode::"TeX:UTF-8"
% !TEX program = xelatex
% texlive2017, xelatex
% author: Ma Seoyin, 2014 Applied Physics, SCUT
% I (Chen Yue, 2019 Mathematics and Applied Mathematics, SCUT) made some tiny adjustments to the original template. 

% 使用“\p{}”建立新的段落,注意在“\p{}”内部也可以分段但在跨页时会遇到问题。

\documentclass{article}
% Allow building from project root or tex/ directory.
\newcommand{\projectroot}{.}
\IfFileExists{styles/template.sty}{}{\renewcommand{\projectroot}{..}}
\usepackage{\projectroot/styles/template}

\usepackage[a4paper,top=3cm, bottom=1.65cm,left=2.54cm,right=2.04cm]{geometry}
\usepackage{amsmath}
\usepackage{graphicx}
\usepackage{amssymb}
\usepackage{amsthm}
\usepackage{tikz-cd}
\usepackage{mathrsfs}
\usepackage[colorinlistoftodos]{todonotes}
\usepackage{yfonts}
\usepackage{ dsfont }
\usepackage{hyperref}
\usepackage{amssymb,amsfonts,amsmath,amsthm}
\usepackage[all,arc]{xy}
\usepackage{enumerate}
\usepackage[mathscr]{eucal}
\usepackage[toc,page]{appendix}
\usepackage{tabularx}
\usepackage{url}
\usepackage{color}
\usepackage{tikz-cd}  % graph
\usepackage{mathdots} %\ddots, \vdots, \iddots
\usepackage{romannum} %Roman number
\usepackage{hyperref} %hyperlink for reference
\usepackage{cite}
\usepackage{tcolorbox}
\usepackage{longtable}


\newtheorem{thm}{Theorem}[section]
\newtheorem{cor}[thm]{Corollary}
\newtheorem{prop}[thm]{Proposition}
\newtheorem{lem}[thm]{Lemma}
\newtheorem{conj}[thm]{Conjecture}
\newtheorem{quest}[thm]{Question}

\theoremstyle{definition}
\newtheorem{defn}[thm]{Definition}
\newtheorem{defns}[thm]{Definitions}
\newtheorem{con}[thm]{Construction}
\newtheorem{exm}[thm]{Example}
\newtheorem{exms}[thm]{Examples}
\newtheorem{notn}[thm]{Notation}
\newtheorem*{notn*}{Notation}
\newtheorem{notns}[thm]{Notations}
\newtheorem{addm}[thm]{Addendum}
\newtheorem{exer}[thm]{Exercise}
\newtheorem{slo}[thm]{Slogan}

\theoremstyle{remark}
\newtheorem{rem}[thm]{Remark}
\newtheorem{rems}[thm]{Remarks}
\newtheorem{warn}[thm]{Warning}
\newtheorem{sch}[thm]{Scholium}
\newtheorem*{idea*}{Idea}


% 定义分段的命令
\newcommand{\p}[1]{\multicolumn{2}{|p{16cm}|}{
#1
} \\}

\graphicspath{{\projectroot/figures/}}
\pagestyle{fancy}
\fancyhf{} % 清空页眉和页脚
\fancyfoot[R]{\thepage} % 设置页码在右下角
\renewcommand{\headrulewidth}{0pt} % 去掉页眉分隔线

\begin{document}
%----------------------------------------------------------------------------------------
%   Cover
%----------------------------------------------------------------------------------------
\thispagestyle{empty}
\begin{figure}[ht]
\centering
\includegraphics[height=2.75cm]{figures/title.png}
\end{figure}
\vspace{1.5cm}

\begin{center}
{\zihao{2}\bfseries 华南理工大学\par}
\vspace{0.6cm}
{\zihao{3}\bfseries 课程论文名称\par}
\vspace{0.8cm}
{\zihao{4}研究生:李XX\par}
\end{center}
\vspace{0.2cm}
{\zihao{5}%
\noindent\begin{tabular*}{\linewidth}{@{}l@{\extracolsep{\fill}}l@{}}
提交日期:\hspace{4em}年\hspace{1em}月\hspace{1em}日 & 研究生签名:\hspace{6em} \\
\end{tabular*}
}
\vspace{0.2cm}

{\zihao{5}%
\def\LTcaptype{none} % do not increment counter
\begin{longtable}[]{@{}|l|l|l|l|@{}}
\toprule\noalign{}
\endhead
\bottomrule\noalign{}
\endlastfoot
\hline
学号 & 2025XXXXXXXXX & 学院 & XX学院 \\
\hline
课程编号 & S0001001 & 课程名称 & XX \\
\hline
学位类别 & □博士 □硕士 & 任课教师 & 张XX \\
\hline
学习形式 & □全日制 □非全日制 & 学期 & 2025-2026学年第一学期 \\
\hline
\multicolumn{4}{|p{0.93\linewidth}|}{%
\textbf{教师评语:}\par\vspace{6cm}} \\
\hline
\multicolumn{4}{|l|}{%
成绩评定: 分 任课教师签名: 年 月 日} \\
\hline
\end{longtable}
}
\pagebreak[4]


%----------------------------------------------------------------------------------------
%   BEGIN TO COUNT THE PAGE NUMBER
%----------------------------------------------------------------------------------------
\setboolean{@twoside}{true}
\zihao{-4}
\setcounter{page}{1}
\pagenumbering{arabic}
%----------------------------------------------------------------------------------------
%   BELOW IS YOUR MAIN TEXT. BEGIN.
%----------------------------------------------------------------------------------------
\begin{flushleft}
\begin{longtable}{|l|l|}

\endfirsthead

\hline
\endhead

\hline
\endfoot

\hline
\endlastfoot

\hline
中文译名 &  \\
\hline
外文原文名 & \\
\hline
外文原文版出处 & \\
\hline
\multicolumn{2}{|p{15.5cm}|}{译文:} \\
\p{
这是一个新的段落. 

分段. 
}
\p{
跨页测试. 
}
\p{
跨页测试. 
}
\p{
跨页测试. 
}
\p{
跨页测试. 
}
\p{
跨页测试. 
}
\p{
跨页测试. 
}
\p{
跨页测试. 
}
\p{
跨页测试. 
}
\p{
跨页测试. 
}
\p{
跨页测试. 
}
\p{
跨页测试. 
}
\p{
跨页测试. 
}
\p{
跨页测试. 
}
\p{
跨页测试. 
}
\p{
跨页测试. 
}
\p{
跨页测试. 
}
\p{
跨页测试. 
}
\p{
跨页测试. 
}
\p{
跨页测试. 
}
\p{
跨页测试. 
}
\p{
跨页测试. 
}
\p{
跨页测试. 
}
\p{
跨页测试. 
}
\p{
跨页测试. 
}
\p{
跨页测试. 
}
\p{
跨页测试. 
}
\p{
跨页测试. 
}
\p{
跨页测试. 
}
\p{
跨页测试. 
}
\p{
跨页测试. 
}
\p{
跨页测试. 
}
\p{
跨页测试. 
}
\p{
跨页测试. 
}
\p{
跨页测试. 
}
\p{
跨页测试. 
}
\p{
跨页测试. 
}
\p{
跨页测试. 
}
\p{
我们记数列$\{a_n\}_{n\geq 0}$的极限收敛为
\[
\lim_{n \to \infty} a_n = a. 
\]
}
\end{longtable}
\end{flushleft}

\end{document}
