\section{相关工作}
本节重点讨论基于深度学习的符号回归方法与基于编辑的生成模型。我们从序列生成机制、搜索策略优化以及离散流匹配原理三个维度展开论述,并在最后指出本文方法的关键差异。

\subsection{基于深度学习的自回归方法}
自回归(Autoregressive, AR)模型是当前深度符号回归的主流范式。其核心思想是将数学表达式树通过前序遍历映射为离散 Token 序列 $\mathbf{z}=(z_1,\ldots,z_T)$,并将符号回归建模为在给定观测数据 $(\mathbf{X}, \mathbf{y})$ 条件下的序列生成问题。

\textbf{序列建模与端到端训练。} Biggio 等人\cite{biggio2021nsr}率先提出这一框架,利用 LSTM 编码器处理输入数据点的特征,通过解码器最大化对数似然来生成表达式。Kamienny 等人\cite{kamienny2022e2e}将该架构升级为大规模 Transformer,并引入专门的数值编码器以处理高精度浮点常量。这类方法通常采用标准交叉熵损失进行训练,即
\[
\max_{\theta}\sum_{(\mathbf{X}, \mathbf{y})}\log p_{\theta}\!\left(\mathbf{z}\mid \mathbf{X}, \mathbf{y}\right),
\]
推理阶段多使用集束搜索(Beam Search)生成候选表达式。

\textbf{搜索策略与强化学习集成。} 为缓解自回归生成的“短视”问题,后续研究将规划与强化学习引入解码过程。Shojaee 等人\cite{shojaee2023planning}提出基于 Transformer 的规划方法,将预训练模型作为策略网络嵌入到蒙特卡洛树搜索(MCTS)中;生成过程中同时利用价值网络评估子树完成后的拟合误差,从而剪枝低潜力分支。Yu 等人\cite{yu2024mdlformer}从信息论视角出发提出基于最小描述长度(MDL)的损失,在优化过程中显式惩罚过长的序列描述,迫使模型在拟合精度与模型复杂度之间寻找平衡。Tian 等人\cite{tian2025interactive}采用离线强化学习框架,通过 Critic 网络评估生成轨迹的长期回报,实现对生成策略的迭代优化。

尽管上述改进提升了搜索效率,AR 模型本质上仍受限于单调生成顺序。一旦在根节点或高层算子生成错误,后续 Token 即使局部最优也难以纠正整体结构偏差,导致其在深层嵌套公式场景下鲁棒性不足。

在 AR 生成框架之外,近年也有工作尝试将在线搜索过程蒸馏进一次前向推断或引入可微结构学习:FormulaGPT 使用强化学习式搜索历史作为预训练语料,将搜索策略蒸馏到 Transformer,并在测试时以 in-context 方式复现并更新搜索过程以兼顾效率与鲁棒性\cite{li2024formulagpt};另一类可微符号回归网络则通过端到端训练协同优化表达式结构与参数,代表性工作如 Lu 等在 Neurocomputing 2025 提出的可微 SR 神经网络\cite{lu2025ddsrnn}。

此外,“LLM 生成候选 + 外部验证/优化”正在成为方程发现的新方向:LLM-SR 结合大模型的科学先验与代码生成能力,通过程序化表达与进化搜索提升方程发现效率\cite{shojaee2025llmsr};与之配套的 LLM-SRBench 强调在大模型时代需要避免对经典基准的记忆性拟合,并以更严格协议评测符号准确率\cite{shojaee2025llmsrbench}。这类工作与本文关注的“从次优表达式出发进行结构纠错与精炼”在问题设定上具有互补性。

\subsection{基于编辑的生成模型与流匹配}
与一次生成的 AR 模型不同,基于编辑(Edit-based)的模型允许对序列进行非单调修改(插入、删除、替换),从而具备纠错与精细化优化的能力。下文统一记编辑操作集合为 $\mathcal{O}=\{\mathrm{ins}, \mathrm{del}, \mathrm{sub}\}$。

\textbf{非自回归编辑模型。} Gu 等人提出的 Levenshtein Transformer\cite{gu2019levenshtein}是该方向的代表性工作。该模型通过删除分类器与插入分类器并行预测,在多轮迭代中先删除不合适的 Token,再预测插入占位符数量并填充具体 Token,从而逐步精炼序列。Zheng 等人\cite{zheng2024smieditor}将该机制迁移至分子生成,提出 Smi-editor,通过片段级编辑实现 SMILES 序列的定向优化。这类方法打破了从左到右的生成限制,但通常依赖复杂的训练策略来模拟编辑轨迹。

\textbf{离散流匹配(Discrete Flow Matching)。} Havasi 等人\cite{havasi2025editflows}提出的 Edit Flows 理论将离散编辑过程纳入流匹配框架。该方法将编辑操作视为状态空间中的概率流,利用 Levenshtein 距离等动态规划算法计算源序列 $\mathbf{z}_0$ 与目标序列 $\mathbf{z}^{\ast}$ 之间的最优对齐路径,从而获得每个时间步 $t$ 的编辑操作分布。训练目标不再是预测下一个 Token,而是最小化预测编辑流与真实编辑流之间的差异,例如
\[
\min_{\theta}\ \mathbb{E}_{t, \mathbf{z}_t}\left[\left\lVert u_{\theta}(\cdot \mid \mathbf{z}_t, \mathbf{X}, \mathbf{y}) - u^{\ast}(\cdot \mid \mathbf{z}_t, \mathbf{z}^{\ast})\right\rVert^{2}\right],
\]
其中 $u_{\theta}$ 表示模型对 $\mathcal{O}$ 的预测分布,$u^{\ast}$ 表示由最优对齐路径定义的目标编辑流。该视角使模型能够将任意次优序列逐步“流”向目标序列。

上述编辑建模与流匹配机制与符号回归的“结构搜索 + 参数微调”需求高度契合:删除操作可去除冗余项,插入操作可补全缺失算子,替换操作可修正错误变量或常数。本文方法在此基础上进一步引入条件编码与迭代式搜索策略,使编辑过程受数据分布强约束并具备更稳定的收敛行为。
