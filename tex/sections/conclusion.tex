\section{结论}
本文基于离散编辑流(Edit Flows)方法,将其应用于符号回归并形成 EditFlow 框架,进而将方程发现重构为数据驱动的迭代编辑过程。通过结合 SetTransformer 对非结构化数据的编码能力与定制化 LLaMA 骨干网络对离散序列的深层理解,EditFlow 实现了从次优表达式向目标形式的逐步“流”动与修正。实验结果表明,该模型不仅具备卓越的结构纠错能力,能从严重噪声中精准恢复物理定律,还可作为强力后处理模块显著缓解传统遗传编程算法的局部最优问题。本研究验证了离散流匹配理论在科学发现领域的潜力,为解决符号回归这一 NP 难问题提供了融合深度学习泛化性与符号推理精确性的全新路径。
