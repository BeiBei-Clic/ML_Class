\subsection*{3.3\quad 机器学习基础}
\subsubsection*{3.3.1\quad 关键对象与术语}
\noindent \textbf{一\quad 任务 数据与样本\quad 基本对象与形式化表示}

\noindent 机器学习研究的直接对象是学习问题的形式化表达,而不是孤立的算法技巧。一个学习系统要成立,首先必须明确三件事。系统要完成的任务是什么,系统将从何种经验中改进,以及用何种性能度量来判定改进是否发生。任务给出输出的语义与形式,经验规定可用于学习的信息来源,性能度量则决定训练与比较模型时好坏的标尺。三者共同构成学习问题的边界条件。边界不清,训练过程即便收敛,也可能是在优化与真实目标不一致的替代目标。

\noindent 在形式化层面,最常用的表达是区分输入与输出。通常用 $X$ 表示输入数据,常见形式是特征矩阵或设计矩阵,用 $y$ 表示输出数据,也就是目标、标签或响应。$X$ 表示预测时可获得的描述信息,$y$ 表示训练阶段用作监督信号而在预测时不可直接使用的目标信息。以房价预测为例,$X$ 可以包含面积、户型、地段、楼龄、周边配套等属性,$y$ 是成交价格。训练阶段模型观察 $(X, y)$ 来学习映射规律。部署阶段模型仅接收新的 $X$,输出预测值 $\hat{y}$。这一区分看似只是符号约定,实质上规定了信息可得性的硬约束。任何预测时无法获得的信息,都不应被纳入 $X$。如果把未来信息或与目标几乎同义的字段混入特征,离线指标往往会异常乐观,而上线表现会迅速变差。

\noindent 在 $X$ 与 $y$ 的表示下,数据集由大量样本组成。样本是学习与评估的基本单位,通常对应 $X$ 的一行,也就是一个输入实例的特征向量。特征对应 $X$ 的一列,表示对所有样本以同一规则测量或抽取的属性。仍以房价为例,每套房子是一条样本,面积、卧室数、是否学区房是不同特征。将数据组织为矩阵并非为了形式美观,而是为了把异质对象统一成可计算的表示,从而使学习一个从 $X$ 到 $y$ 的映射成为明确的数学问题。

\noindent 监督学习中的标签应被视为一种观测结果,而不是天然真值。很多任务的标签来自人工标注、业务规则或延迟反馈,因此不可避免地包含噪声、口径不一致以及随时间变化的漂移。以垃圾信息识别为例,$X$ 可由邮件正文、标题、发件域名、历史交互等构成,$y$ 为垃圾或非垃圾。对于边界邮件,不同标注者或不同时期的策略可能给出不同结论。这类不一致会直接体现在模型可达到的上限与误差形态之中。因此,标签定义的清晰性、稳定性与可复现性,是模型性能与实验可信度的前置条件,而不是训练之后才考虑的附属问题。

\noindent 训练与评估的分离是学习问题成立的基本要求。用同一批数据既训练又评估,会把对已见样本的拟合误当作对未见样本的能力,从而得到系统性偏高的结论。实践中通常将数据划分为训练集、验证集与测试集。训练集用于拟合模型参数,验证集用于模型选择与超参数调节,测试集用于对最终模型给出尽可能无偏的泛化估计。需要强调的是,分离不仅发生在模型训练层面,也必须贯穿所有会从数据中估计统计量的步骤。以标准化为例,如果先用全量数据计算均值与方差再切分数据,测试集的信息已经通过统计量渗入训练过程。类似风险同样存在于缺失值填充、特征选择、降维、目标编码等预处理之中。可靠的流程是先切分,再在训练数据上拟合预处理参数,并把同样的变换仅应用到验证集与测试集上。

\noindent 此外,样本之间并不总是相互独立。现实数据常存在同源相关性,例如同一用户的多次行为、同一设备的连续监测窗口、同一患者的多次影像检查等。如果这些高度相关的样本同时出现在训练集与测试集,模型可能通过记住实体特征取得高分,而在真正的新实体上表现明显下降。以医疗影像为例,如果同一患者的多张 X 光片分散到训练与测试中,模型可能利用与病灶无关但与患者个体相关的低层纹理或成像条件获得虚高的测试成绩。因而,数据划分策略应与泛化目标一致。当目标是对新用户、新设备或新时间段泛化时,划分就应避免相关样本跨集合混入,从评估机制上保证测试集确实代表未见条件。

\noindent 通过上述概念体系,学习问题获得了统一而严格的表述。用 $X$ 描述可用输入,用 $y$ 定义监督目标,以样本与特征的矩阵结构组织数据,并通过训练、验证、测试的隔离流程约束评估可信度,同时在必要时显式处理样本相关结构对泛化判断的影响。后续关于模型、损失函数与优化方法的讨论,均以这些基本对象与表示为基础展开。
\subsubsection*{3.3.2\quad 监督学习}
\subsubsection*{3.3.3\quad 无监督学习}
\subsubsection*{3.3.4\quad 强化学习}
\subsubsection*{3.3.5\quad 模型训练与评估}
\subsubsection*{3.3.6\quad 常用算法概览}
\paragraph{3.3.6.1\quad 回归方法}
\paragraph{3.3.6.2\quad 分类方法}
\paragraph{3.3.6.3\quad 聚类方法}
