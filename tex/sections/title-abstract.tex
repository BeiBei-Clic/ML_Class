\section{摘要}
随着数据驱动方法在科学发现中的广泛应用,符号回归因能够从数据中直接反演解析表达式而受到关注,但其搜索空间庞大、结构与参数高度耦合,导致求解效率与稳定性受限。本文提出 EditFlow,将符号回归重构为以观测数据为条件的离散编辑流过程,通过 SetTransformer 对无序样本进行编码,并在 LLaMA 骨干中注入交叉注意力,实现对表达式序列的多位置编辑预测。训练阶段引入随机化对齐与流匹配损失,缓解离散编辑路径的不唯一性;推理阶段采用迭代贪婪编辑并结合常数优化。基于 Feynman 基准的实验表明,EditFlow 能够从受损表达式中稳健恢复物理定律,并在与 PySR 的协同策略中显著提升次优解质量。研究结果验证了离散流匹配在符号回归中的潜力,为高可解释模型发现提供了新路径。

\textbf{关键词:} 符号回归;离散编辑流;SetTransformer;LLaMA
