\section{实验结果}
为了验证 EditFlow 在符号回归任务中的有效性,我们在 Feynman Symbolic Regression Benchmark 数据集上进行了评估。本节聚焦两类关键场景:(1) 对受损物理方程的结构修复能力;(2) 作为后处理模块与遗传编程方法(PySR\cite{cranmer2023pysr})的协同优化能力。

\subsection{物理定律的编辑与修复能力}
EditFlow 的核心假设是:通过学习离散编辑流,模型能够根据数据分布将错误表达式逐步“推”向正确形式。为验证该能力,我们选取若干经典物理方程,并人为引入结构噪声(算子错误、变量缺失或冗余项)。将受损表达式作为推理初始状态 $\mathbf{z}_0$,观察模型是否能够修复为目标方程。

\begin{table}[t]
\centering
\caption{EditFlow 对受损物理方程的逐步修复过程示例}
\begin{tabularx}{\linewidth}{l l l l X l l}
\toprule
目标方程 & 含义 & 初始受损状态 $\mathbf{z}_0$ & 错误类型 & 修复路径 & 最终恢复结果 & 状态 \\
\midrule
$E = m c^{2}$ & 质能方程 & $E = m + c^{2}$ & 算子错误 & $m + c^{2} \rightarrow m c^{2}$ & $E = m c^{2}$ & $\checkmark$ \\
$F = G \frac{m_1 m_2}{r^{2}}$ & 万有引力 & $F = G \frac{m_1}{r^{2}}$ & 变量缺失 & 插入 $m_2$ 于分子 & $F = G \frac{m_1 m_2}{r^{2}}$ & $\checkmark$ \\
$PV = nRT$ & 理想气体 & $PV = nRT + R$ & 冗余项 & 删除 $+R$ & $PV = nRT$ & $\checkmark$ \\
$h = \frac{1}{2} g t^{2}$ & 自由落体 & $h = g t$ & 结构缺失 & 插入 $\frac{1}{2}$ 与 $t^{2}$ & $h = \frac{1}{2} g t^{2}$ & $\checkmark$ \\
$x = v_0 t + \frac{1}{2} a t^{2}$ & 匀加速运动 & $x = v_0 t - \frac{1}{2} a t^{2}$ & 算子错误 & 替换 $-$ 为 $+$ & $x = v_0 t + \frac{1}{2} a t^{2}$ & $\checkmark$ \\
\bottomrule
\end{tabularx}
\end{table}

如表 1 所示,EditFlow 展现出较强的语义纠错能力。即使初始表达式与目标存在较大结构偏差(如万有引力公式中缺失关键变量 $m_2$),模型仍可利用 SetTransformer 从观测数据 $(\mathbf{X}, \mathbf{y})$ 中提取的条件特征,精准预测插入或替换操作方向,从而高效逼近正确方程。

\subsection{与遗传编程的协同优化}
遗传编程(如 PySR)在符号回归中表现优异,但在高维或复杂函数上常受限于搜索时间,容易陷入局部最优。为此,我们提出“PySR + EditFlow”的混合策略:首先运行 PySR 进行短时搜索(迭代预算为标准设置的 10\%),再将其得到的次优解输入 EditFlow 进行精炼。

\begin{table}[t]
\centering
\caption{混合策略(PySR + EditFlow)与基线方法对比}
\begin{tabularx}{\linewidth}{l c X c X c}
\toprule
目标方程 & 复杂度 & PySR (Early Stop) 输出 & Score & PySR + EditFlow 输出 & Score \\
\midrule
$y = x^{3} + x$ & 低 & $y = x^{2} + x$ & 0.892 & $y = x^{3} + x$ & 0.999 \\
$y = \sin(x_{1}) + x_{2}^{2}$ & 中 & $y = \sin(x_{1})$ & 0.753 & $y = \sin(x_{1}) + x_{2}^{2}$ & 0.998 \\
$y = \frac{x_{1} x_{2}}{x_{3} + 1}$ & 高 & $y = x_{1} x_{2}$ & 0.621 & $y = \frac{x_{1} x_{2}}{x_{3} + 1}$ & 0.995 \\
$y = \exp(x)$ & 中 & $y = 1 + x + \frac{1}{2}x^{2}$ & 0.810 & $y = \exp(x)$ & 0.999 \\
\bottomrule
\end{tabularx}
\end{table}

表 2 的结果表明,EditFlow 能够作为传统符号回归算法的高效后处理器:一方面帮助模型跳出局部最优(如将二次近似修正为三次结构),另一方面补全被搜索过程遗漏的关键项或分母结构。该混合策略以较低的额外计算成本显著提升最终精度,具备良好的工程实用性。
